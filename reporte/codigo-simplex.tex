% Principio
\begin{tcolorbox}[breakable, size=fbox, boxrule=1pt, pad at break*=1mm,colback=cellbackground, colframe=cellborder]
\prompt{In}{incolor}{1}{\boxspacing}
\begin{Verbatim}[commandchars=\\\{\}]
\PY{k+kn}{import} \PY{n+nn}{numpy} \PY{k}{as} \PY{n+nn}{np}
\PY{n}{np}\PY{o}{.}\PY{n}{set\PYZus{}printoptions}\PY{p}{(}\PY{n}{formatter}\PY{o}{=}\PY{p}{\PYZob{}}\PY{l+s+s1}{\PYZsq{}}\PY{l+s+s1}{float}\PY{l+s+s1}{\PYZsq{}}\PY{p}{:} \PY{k}{lambda} \PY{n}{x}\PY{p}{:} \PY{l+s+s2}{\PYZdq{}}\PY{l+s+si}{\PYZob{}0:0.3f\PYZcb{}}\PY{l+s+s2}{\PYZdq{}}\PY{o}{.}\PY{n}{format}\PY{p}{(}\PY{n}{x}\PY{p}{)}\PY{p}{\PYZcb{}}\PY{p}{)} \PY{c+c1}{\PYZsh{} Para que enseñe nada más 3 decimales}
\end{Verbatim}
\end{tcolorbox}

% Función matriz_t0:
\begin{tcolorbox}[breakable, size=fbox, boxrule=1pt, pad at break*=1mm,colback=cellbackground, colframe=cellborder]
\prompt{In}{incolor}{2}{\boxspacing}
\begin{Verbatim}[commandchars=\\\{\}]
\PY{k}{def} \PY{n+nf}{matriz\PYZus{}t0}\PY{p}{(}\PY{n}{A}\PY{p}{,} \PY{n}{b}\PY{p}{,} \PY{n}{c}\PY{p}{,} \PY{n}{M}\PY{p}{)}\PY{p}{:}
    \PY{l+s+sd}{\PYZsq{}\PYZsq{}\PYZsq{}}
\PY{l+s+sd}{    Método para obtener la tabla cero del problema usando el método}
\PY{l+s+sd}{    de la Gran M.}

\PY{l+s+sd}{    EJEMPLO DE USO:}
\PY{l+s+sd}{    \PYZgt{}\PYZgt{}\PYZgt{} A = np.array([[3, 4, 1, 0],}
\PY{l+s+sd}{                     [ 2,\PYZhy{}1, 0,\PYZhy{}1]])}
\PY{l+s+sd}{    \PYZgt{}\PYZgt{}\PYZgt{} b = np.array([20,2])}
\PY{l+s+sd}{    \PYZgt{}\PYZgt{}\PYZgt{} c = np.array([1, 1, 0, 0])}
\PY{l+s+sd}{    \PYZgt{}\PYZgt{}\PYZgt{} M = 100}
\PY{l+s+sd}{    \PYZgt{}\PYZgt{}\PYZgt{} matriz\PYZus{}t0(A, b, c, M)}
\PY{l+s+sd}{    array([[  3.,   4.,   1.,   0.,   1.,   0.,  20.],}
\PY{l+s+sd}{           [  2.,  \PYZhy{}1.,   0.,  \PYZhy{}1.,   0.,   1.,   2.],}
\PY{l+s+sd}{           [  1.,   1.,   0.,   0., 100., 100.,   0.]])  }
\PY{l+s+sd}{    \PYZsq{}\PYZsq{}\PYZsq{}}
    \PY{n}{m} \PY{o}{=} \PY{n+nb}{len}\PY{p}{(}\PY{n}{A}\PY{p}{)}
    \PY{n}{canon} \PY{o}{=} \PY{n}{np}\PY{o}{.}\PY{n}{eye}\PY{p}{(}\PY{n}{m}\PY{p}{,} \PY{n}{dtype}\PY{o}{=}\PY{n+nb}{int}\PY{p}{)}                     \PY{c+c1}{\PYZsh{} Matriz identidad mxm}
    \PY{n}{mat1} \PY{o}{=} \PY{n}{np}\PY{o}{.}\PY{n}{concatenate}\PY{p}{(}\PY{p}{(}\PY{n}{A}\PY{p}{,}\PY{n}{canon}\PY{p}{)}\PY{p}{,} \PY{n}{axis}\PY{o}{=}\PY{l+m+mi}{1}\PY{p}{)}         \PY{c+c1}{\PYZsh{} Pegamos A con la identidad}
    \PY{n}{cr} \PY{o}{=} \PY{n}{np}\PY{o}{.}\PY{n}{append}\PY{p}{(}\PY{n}{c}\PY{p}{,}\PY{p}{[}\PY{n}{M}\PY{p}{]}\PY{o}{*}\PY{n}{m}\PY{p}{)}                          \PY{c+c1}{\PYZsh{} Vector de costos relativos}
    \PY{n}{mat1} \PY{o}{=} \PY{n}{np}\PY{o}{.}\PY{n}{concatenate}\PY{p}{(}\PY{p}{(}\PY{n}{mat1}\PY{p}{,} \PY{n}{np}\PY{o}{.}\PY{n}{array}\PY{p}{(}\PY{p}{[}\PY{n}{cr}\PY{p}{]}\PY{p}{)}\PY{p}{)}\PY{p}{)}    \PY{c+c1}{\PYZsh{} Pegamos la matriz con los costos relativos}
    \PY{n}{b\PYZus{}ext} \PY{o}{=} \PY{n}{np}\PY{o}{.}\PY{n}{array}\PY{p}{(}\PY{p}{[}\PY{n}{np}\PY{o}{.}\PY{n}{append}\PY{p}{(}\PY{n}{b}\PY{p}{,} \PY{l+m+mi}{0}\PY{p}{)}\PY{p}{]}\PY{p}{)}              \PY{c+c1}{\PYZsh{} Construcción del vector b}
    
    \PY{c+c1}{\PYZsh{} Regresamos la matriz extendida con b}
    \PY{k}{return} \PY{n}{np}\PY{o}{.}\PY{n}{concatenate}\PY{p}{(}\PY{p}{(}\PY{n}{mat1}\PY{p}{,} \PY{n}{b\PYZus{}ext}\PY{o}{.}\PY{n}{T}\PY{p}{)}\PY{p}{,} \PY{n}{axis}\PY{o}{=}\PY{l+m+mi}{1}\PY{p}{)}\PY{o}{.}\PY{n}{astype}\PY{p}{(}\PY{l+s+s1}{\PYZsq{}}\PY{l+s+s1}{float64}\PY{l+s+s1}{\PYZsq{}}\PY{p}{)}
\end{Verbatim}
\end{tcolorbox}

% Función reglaDeBland:
\begin{tcolorbox}[breakable, size=fbox, boxrule=1pt, pad at break*=1mm,colback=cellbackground, colframe=cellborder]
\prompt{In}{incolor}{3}{\boxspacing}
\begin{Verbatim}[commandchars=\\\{\}]
\PY{k}{def} \PY{n+nf}{reglaDeBland}\PY{p}{(}\PY{n}{tablaSimplex}\PY{p}{)}\PY{p}{:}
    \PY{l+s+sd}{\PYZsq{}\PYZsq{}\PYZsq{}}
\PY{l+s+sd}{    Queremos la columna más a la izquierda con cr \PYZlt{} 0.}
\PY{l+s+sd}{    Si hay empate en el criterio de la variable de salida,}
\PY{l+s+sd}{    elegimos la más arriba}
\PY{l+s+sd}{    \PYZsq{}\PYZsq{}\PYZsq{}}
    \PY{n}{cr} \PY{o}{=} \PY{n}{tablaSimplex}\PY{p}{[}\PY{o}{\PYZhy{}}\PY{l+m+mi}{1}\PY{p}{]}\PY{p}{[}\PY{p}{:}\PY{o}{\PYZhy{}}\PY{l+m+mi}{1}\PY{p}{]} \PY{c+c1}{\PYZsh{} costos relativos}
    \PY{n}{busq} \PY{o}{=} \PY{n}{np}\PY{o}{.}\PY{n}{where}\PY{p}{(}\PY{n}{cr} \PY{o}{\PYZlt{}} \PY{l+m+mi}{0}\PY{p}{)}\PY{p}{[}\PY{l+m+mi}{0}\PY{p}{]}
    
    \PY{k}{if} \PY{n+nb}{len}\PY{p}{(}\PY{n}{busq}\PY{p}{)} \PY{o}{==} \PY{l+m+mi}{0}\PY{p}{:} \PY{c+c1}{\PYZsh{} No encontró; fin del problema}
        \PY{k}{return} \PY{o}{\PYZhy{}}\PY{l+m+mi}{1}\PY{p}{,} \PY{o}{\PYZhy{}}\PY{l+m+mi}{1}
    \PY{k}{else}\PY{p}{:}
        \PY{n}{colSal} \PY{o}{=} \PY{n}{busq}\PY{p}{[}\PY{l+m+mi}{0}\PY{p}{]}
        
    \PY{n}{bk} \PY{o}{=} \PY{n}{tablaSimplex}\PY{p}{[}\PY{p}{:}\PY{p}{,} \PY{o}{\PYZhy{}}\PY{l+m+mi}{1}\PY{p}{]}
    \PY{n}{yk} \PY{o}{=} \PY{n}{tablaSimplex}\PY{p}{[}\PY{p}{:}\PY{p}{,} \PY{n}{colSal}\PY{p}{]}
    
    \PY{n}{by} \PY{o}{=} \PY{n}{np}\PY{o}{.}\PY{n}{empty}\PY{p}{(}\PY{l+m+mi}{0}\PY{p}{)}
    \PY{k}{for} \PY{n}{b}\PY{p}{,}\PY{n}{y} \PY{o+ow}{in} \PY{n+nb}{zip}\PY{p}{(}\PY{n}{bk}\PY{p}{,}\PY{n}{yk}\PY{p}{)}\PY{p}{:}
        \PY{k}{if} \PY{n}{y} \PY{o}{\PYZgt{}} \PY{l+m+mi}{0}\PY{p}{:}
            \PY{n}{by} \PY{o}{=} \PY{n}{np}\PY{o}{.}\PY{n}{append}\PY{p}{(}\PY{n}{by}\PY{p}{,} \PY{n}{b}\PY{o}{/}\PY{n}{y}\PY{p}{)}
        \PY{k}{else}\PY{p}{:}
            \PY{n}{by} \PY{o}{=} \PY{n}{np}\PY{o}{.}\PY{n}{append}\PY{p}{(}\PY{n}{by}\PY{p}{,} \PY{o}{\PYZhy{}}\PY{l+m+mi}{1}\PY{p}{)}
    
    \PY{n}{valid} \PY{o}{=} \PY{n}{np}\PY{o}{.}\PY{n}{where}\PY{p}{(}\PY{n}{by} \PY{o}{\PYZgt{}}\PY{o}{=} \PY{l+m+mi}{0}\PY{p}{)}\PY{p}{[}\PY{l+m+mi}{0}\PY{p}{]}
    
    \PY{k}{if} \PY{n+nb}{len}\PY{p}{(}\PY{n}{valid}\PY{p}{)} \PY{o}{==} \PY{l+m+mi}{0}\PY{p}{:} \PY{c+c1}{\PYZsh{} Todas las variables son menores que cero}
        \PY{k}{return} \PY{o}{\PYZhy{}}\PY{l+m+mi}{1}\PY{p}{,} \PY{o}{\PYZhy{}}\PY{l+m+mi}{2}
    
    \PY{n}{renglonSal} \PY{o}{=} \PY{n}{valid}\PY{p}{[}\PY{n}{by}\PY{p}{[}\PY{n}{valid}\PY{p}{]}\PY{o}{.}\PY{n}{argmin}\PY{p}{(}\PY{p}{)}\PY{p}{]}
    
    \PY{k}{return} \PY{n}{renglonSal}\PY{p}{,} \PY{n}{colSal}
\end{Verbatim}
\end{tcolorbox}

% Función pivoteo:
\begin{tcolorbox}[breakable, size=fbox, boxrule=1pt, pad at break*=1mm,colback=cellbackground, colframe=cellborder]
\prompt{In}{incolor}{4}{\boxspacing}
\begin{Verbatim}[commandchars=\\\{\}]
\PY{k}{def} \PY{n+nf}{pivoteo}\PY{p}{(}\PY{n}{tablaSimplex}\PY{p}{)}\PY{p}{:}
    \PY{l+s+sd}{\PYZsq{}\PYZsq{}\PYZsq{}}
\PY{l+s+sd}{    Dada una tabla Simplex, este método pivotea sobre el elemento }
\PY{l+s+sd}{    que dictamina la regla de Bland y regresa el resultado.}
\PY{l+s+sd}{    \PYZsq{}\PYZsq{}\PYZsq{}}
    \PY{n}{renglonSal}\PY{p}{,} \PY{n}{colSal} \PY{o}{=} \PY{n}{reglaDeBland}\PY{p}{(}\PY{n}{tablaSimplex}\PY{p}{)}
    
    \PY{k}{if} \PY{n}{colSal} \PY{o}{\PYZlt{}} \PY{l+m+mi}{0}\PY{p}{:} \PY{c+c1}{\PYZsh{} Condiciones para detenerse}
        \PY{k}{return} \PY{n}{tablaSimplex}\PY{p}{,} \PY{n}{colSal}
    
    \PY{n}{m} \PY{o}{=} \PY{n+nb}{len}\PY{p}{(}\PY{n}{tablaSimplex}\PY{p}{)}
    
    \PY{n}{valorPivoteo} \PY{o}{=} \PY{n}{tablaSimplex}\PY{p}{[}\PY{n}{renglonSal}\PY{p}{]}\PY{p}{[}\PY{n}{colSal}\PY{p}{]}
    \PY{n}{tablaSimplex}\PY{p}{[}\PY{n}{renglonSal}\PY{p}{]} \PY{o}{=} \PY{n}{tablaSimplex}\PY{p}{[}\PY{n}{renglonSal}\PY{p}{]} \PY{o}{/} \PY{n}{valorPivoteo}
    
    \PY{k}{for} \PY{n}{i} \PY{o+ow}{in} \PY{n+nb}{range}\PY{p}{(}\PY{n}{m}\PY{p}{)}\PY{p}{:}
        \PY{k}{if} \PY{n}{i} \PY{o}{!=} \PY{n}{renglonSal} \PY{o+ow}{and} \PY{n}{tablaSimplex}\PY{p}{[}\PY{n}{i}\PY{p}{]}\PY{p}{[}\PY{n}{colSal}\PY{p}{]} \PY{o}{!=} \PY{l+m+mi}{0}\PY{p}{:}
            \PY{n}{tablaSimplex}\PY{p}{[}\PY{n}{i}\PY{p}{]} \PY{o}{\PYZhy{}}\PY{o}{=} \PY{n}{tablaSimplex}\PY{p}{[}\PY{n}{i}\PY{p}{]}\PY{p}{[}\PY{n}{colSal}\PY{p}{]} \PY{o}{*} \PY{n}{tablaSimplex}\PY{p}{[}\PY{n}{renglonSal}\PY{p}{]}
    
    \PY{k}{return} \PY{n}{tablaSimplex}\PY{p}{,} \PY{l+m+mi}{1}
\end{Verbatim}
\end{tcolorbox}

% Función solver:
\begin{tcolorbox}[breakable, size=fbox, boxrule=1pt, pad at break*=1mm,colback=cellbackground, colframe=cellborder]
\prompt{In}{incolor}{5}{\boxspacing}
\begin{Verbatim}[commandchars=\\\{\}]
\PY{k}{def} \PY{n+nf}{solver}\PY{p}{(}\PY{n}{A}\PY{p}{,}\PY{n}{b}\PY{p}{,}\PY{n}{c}\PY{p}{,}\PY{n}{M}\PY{o}{=}\PY{l+m+mi}{100}\PY{p}{)}\PY{p}{:}
    \PY{l+s+sd}{\PYZsq{}\PYZsq{}\PYZsq{}}
\PY{l+s+sd}{    Método para resolver un PPL planteado en su forma estándar. Utiliza}
\PY{l+s+sd}{    el método de la Gran M y Simplex. Regresa la tabla en su forma final}
\PY{l+s+sd}{    y el resultado de la función objetivo.}
\PY{l+s+sd}{    \PYZsq{}\PYZsq{}\PYZsq{}}
    \PY{n}{t0} \PY{o}{=}  \PY{n}{matriz\PYZus{}t0}\PY{p}{(}\PY{n}{A}\PY{p}{,} \PY{n}{b}\PY{p}{,} \PY{n}{c}\PY{p}{,}\PY{n}{M}\PY{p}{)}
    
    \PY{k}{for} \PY{n}{i} \PY{o+ow}{in} \PY{n+nb}{range}\PY{p}{(}\PY{n+nb}{len}\PY{p}{(}\PY{n}{A}\PY{p}{)}\PY{p}{)}\PY{p}{:}
        \PY{n}{t0}\PY{p}{[}\PY{o}{\PYZhy{}}\PY{l+m+mi}{1}\PY{p}{]} \PY{o}{=} \PY{n}{t0}\PY{p}{[}\PY{n}{i}\PY{p}{]}\PY{o}{*}\PY{p}{(}\PY{o}{\PYZhy{}}\PY{n}{M}\PY{p}{)} \PY{o}{+} \PY{n}{t0}\PY{p}{[}\PY{o}{\PYZhy{}}\PY{l+m+mi}{1}\PY{p}{]}
    
    \PY{n+nb}{print}\PY{p}{(}\PY{n}{t0}\PY{p}{)}
    \PY{n+nb}{print}\PY{p}{(}\PY{l+s+s1}{\PYZsq{}}\PY{l+s+s1}{\PYZsq{}}\PY{p}{)}
    
    \PY{n}{t1}\PY{p}{,} \PY{n}{z} \PY{o}{=} \PY{n}{pivoteo}\PY{p}{(}\PY{n}{t0}\PY{p}{)} \PY{c+c1}{\PYZsh{} Aquí z es la columna de salida y la usamos como control para saber si terminó.}
    
    \PY{k}{while} \PY{n}{z} \PY{o}{\PYZgt{}}\PY{o}{=} \PY{l+m+mi}{0}\PY{p}{:}
        \PY{n}{t1}\PY{p}{,} \PY{n}{z} \PY{o}{=} \PY{n}{pivoteo}\PY{p}{(}\PY{n}{t1}\PY{p}{)}
    
    \PY{k}{if} \PY{n}{z} \PY{o}{==} \PY{o}{\PYZhy{}}\PY{l+m+mi}{2}\PY{p}{:} \PY{c+c1}{\PYZsh{} no está acotado}
        \PY{n+nb}{print}\PY{p}{(}\PY{l+s+s2}{\PYZdq{}}\PY{l+s+s2}{PROBLEMA NO ACOTADO; última versión de la tabla:}\PY{l+s+s2}{\PYZdq{}}\PY{p}{)}
        \PY{k}{return} \PY{n}{t1}\PY{p}{,} \PY{n}{np}\PY{o}{.}\PY{n}{nan}
    
    \PY{k}{return} \PY{n}{t1}\PY{p}{,} \PY{p}{(}\PY{o}{\PYZhy{}}\PY{l+m+mi}{1}\PY{p}{)}\PY{o}{*}\PY{n}{t1}\PY{p}{[}\PY{o}{\PYZhy{}}\PY{l+m+mi}{1}\PY{p}{]}\PY{p}{[}\PY{o}{\PYZhy{}}\PY{l+m+mi}{1}\PY{p}{]}
\end{Verbatim}
\end{tcolorbox}