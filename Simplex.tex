\hypertarget{proyecto-de-programaciuxf3n-lineal}{%
\section{Proyecto de Programación
Lineal}\label{proyecto-de-programaciuxf3n-lineal}}

\hypertarget{programaciuxf3n-del-muxe9todo-simplex}{%
\subsection{Programación del Método
Simplex}\label{programaciuxf3n-del-muxe9todo-simplex}}

\begin{lstlisting}[language=Python]
import numpy as np
np.set_printoptions(formatter={'float': lambda x: "{0:0.3f}".format(x)}) # Para que enseñe nada más 3 decimales
\end{lstlisting}

\begin{lstlisting}[language=Python]
def matriz_t0(A, b, c, M):
    '''
    Método para obtener la tabla cero del problema usando el método
    de la Gran M.

    EJEMPLO DE USO:
    >>> A = np.array([[3, 4, 1, 0],
                     [ 2,-1, 0,-1]])
    >>> b = np.array([20,2])
    >>> c = np.array([1, 1, 0, 0])
    >>> M = 100
    >>> matriz_t0(A, b, c, M)
    array([[  3.,   4.,   1.,   0.,   1.,   0.,  20.],
           [  2.,  -1.,   0.,  -1.,   0.,   1.,   2.],                                                                             [  1.,   1.,   0.,   0., 100., 100.,   0.]])  
    '''
    m = len(A)
    canon = np.eye(m, dtype=int)                     # Matriz identidad mxm
    mat1 = np.concatenate((A,canon), axis=1)         # Pegamos A con la identidad
    cr = np.append(c,[M]*m)                          # Vector de costos relativos
    mat1 = np.concatenate((mat1, np.array([cr])))    # Pegamos la matriz con los costos relativos
    b_ext = np.array([np.append(b, 0)])              # Construcción del vector b
    
    # Regresamos la matriz extendida con b
    return np.concatenate((mat1, b_ext.T), axis=1).astype('float64')
\end{lstlisting}

\begin{lstlisting}[language=Python]
def reglaDeBland(tablaSimplex):
    '''
    Queremos la columna más a la izquierda con cr < 0.
    Si hay empate en el criterio de la variable de salida,
    elegimos la más arriba
    '''
    cr = tablaSimplex[-1][:-1] # costos relativos
    busq = np.where(cr < 0)[0]
    
    if len(busq) == 0: # No encontró; fin del problema
        return -1, -1
    else:
        colSal = busq[0]
        
    bk = tablaSimplex[:, -1]
    yk = tablaSimplex[:, colSal]
    
    by = np.empty(0)
    for b,y in zip(bk,yk):
        if y > 0:
            by = np.append(by, b/y)
        else:
            by = np.append(by, -1)
    
    valid = np.where(by >= 0)[0]
    
    if len(valid) == 0: # Todas las variables son menores que cero
        return -1, -2
    
    renglonSal = valid[by[valid].argmin()]
    
    return renglonSal, colSal
\end{lstlisting}

\begin{lstlisting}[language=Python]
def pivoteo(tablaSimplex):
    '''
    Dada una tabla Simplex, este método pivotea sobre el elemento 
    que dictamina la regla de Bland y regresa el resultado.
    '''
    renglonSal, colSal = reglaDeBland(tablaSimplex)
    
    if colSal < 0: # Condiciones para detenerse
        return tablaSimplex, colSal
    
    m = len(tablaSimplex)
    
    valorPivoteo = tablaSimplex[renglonSal][colSal]
    tablaSimplex[renglonSal] = tablaSimplex[renglonSal] / valorPivoteo
    
    for i in range(m):
        if i != renglonSal and tablaSimplex[i][colSal] != 0:
            tablaSimplex[i] -= tablaSimplex[i][colSal] * tablaSimplex[renglonSal]
    
    return tablaSimplex, 1
\end{lstlisting}

\begin{lstlisting}[language=Python]
def solver(A,b,c,M=100):
    '''
    Método para resolver un PPL planteado en su forma estándar. Utiliza
    el método de la Gran M y Simplex. Regresa la tabla en su forma final
    y el resultado de la función objetivo.
    '''
    t0 =  matriz_t0(A, b, c,M)
    
    for i in range(len(A)):
        t0[-1] = t0[i]*(-M) + t0[-1]
    
    print(t0)
    print('')
    
    t1, z = pivoteo(t0) # Aquí z es la columna de salida y la usamos como control para saber si terminó.
    
    while z >= 0:
        t1, z = pivoteo(t1)
    
    if z == -2: # no está acotado
        print("PROBLEMA NO ACOTADO; última versión de la tabla:")
        return t1, np.nan
    
    return t1, (-1)*t1[-1][-1]
\end{lstlisting}

\begin{lstlisting}[language=Python]
c = np.array([1, 1, 0, 0])
A = np.array([[3, 4, 1, 0],
            [ 2,-1, 0,-1]])
b = np.array([20,2])
M = 100

t1, z = solver(A, b, c, M)
print(t1)
print(f"El valor de la función objetivo es: {z}")
\end{lstlisting}

\begin{lstlisting}
[[3.0 4.0 1.0 0.0 1.0 0.0 20.0]
 [2.0 -1.0 0.0 -1.0 0.0 1.0 2.0]
 [-499.0 -299.0 -100.0 100.0 0.0 0.0 -2200.0]]

[[0.0 5.5 1.0 1.5 1.0 -1.5 17.0]
 [1.0 -0.5 0.0 -0.5 0.0 0.5 1.0]
 [0.0 1.5 0.0 0.5 100.0 99.5 -1.0]]
El valor de la función objetivo es: 1.0000000000002132
\end{lstlisting}

\begin{lstlisting}[language=Python]
c1 = np.array([0, -9, -1, 0, 2, 1])
A1 = np.array([[0, 5, 50, 1, 1, 0],
               [1, -15, 2, 0, 0, 0],
               [0, 1, 1, 0, 1, 1]])
b1 = np.array([10, 2, 6])
t1, z1 = solver(A1, b1, c1, M)
print(t1)
print(f"\nEl valor de la función objetivo es: {z1}")
\end{lstlisting}

\begin{lstlisting}
[[0.0 5.0 50.0 1.0 1.0 0.0 1.0 0.0 0.0 10.0]
 [1.0 -15.0 2.0 0.0 0.0 0.0 0.0 1.0 0.0 2.0]
 [0.0 1.0 1.0 0.0 1.0 1.0 0.0 0.0 1.0 6.0]
 [-100.0 891.0 -5301.0 -100.0 -198.0 -99.0 0.0 0.0 0.0 -1800.0]]

[[0.0 1.0 10.0 0.2 0.2 0.0 0.2 0.0 0.0 2.0]
 [1.0 0.0 152.0 3.0 3.0 0.0 3.0 1.0 0.0 32.0]
 [0.0 0.0 -9.0 -0.2 0.8 1.0 -0.2 0.0 1.0 4.0]
 [0.0 0.0 98.0 2.0 3.0 0.0 102.0 100.0 99.0 14.0]]

El valor de la función objetivo es: -14.0
\end{lstlisting}

\begin{lstlisting}[language=Python]
c2 = np.array([-3, 1, 0, 0])
A2 = np.array([[-1, 1, 1, 0],
            [2, 2, 0, -1]])
b2 = np.array([5, 4])
t2, z2 = solver(A2, b2, c2, M)
print(t2)
print(f"\nEl valor de la función objetivo es: {z2}")
\end{lstlisting}

\begin{lstlisting}
[[-1.0 1.0 1.0 0.0 1.0 0.0 5.0]
 [2.0 2.0 0.0 -1.0 0.0 1.0 4.0]
 [-103.0 -299.0 -100.0 100.0 0.0 0.0 -900.0]]

PROBLEMA NO ACOTADO; última versión de la tabla:
[[0.0 2.0 1.0 -0.5 1.0 0.5 7.0]
 [1.0 1.0 0.0 -0.5 0.0 0.5 2.0]
 [0.0 4.0 0.0 -1.5 100.0 101.5 6.0]]

El valor de la función objetivo es: nan
\end{lstlisting}

\begin{lstlisting}[language=Python]
c3 = np.array([-40, -30, 0, 0])
A3 = np.array([[1, 1, 1, 0],
              [2, 1, 0, 1]])
b3 = np.array([12, 16])
t3, z3 = solver(A3, b3, c3, M)
print(t3)
print(f"\nEl valor de la función objetivo es: {z3}")
\end{lstlisting}

\begin{lstlisting}
[[1.0 1.0 1.0 0.0 1.0 0.0 12.0]
 [2.0 1.0 0.0 1.0 0.0 1.0 16.0]
 [-340.0 -230.0 -100.0 -100.0 0.0 0.0 -2800.0]]

[[0.0 1.0 2.0 -1.0 2.0 -1.0 8.0]
 [1.0 0.0 -1.0 1.0 -1.0 1.0 4.0]
 [0.0 0.0 20.0 10.0 120.0 110.0 400.0]]

El valor de la función objetivo es: -400.0
\end{lstlisting}

\begin{lstlisting}[language=Python]
\end{lstlisting}
